\chapter{Results}
\section{Overview}
Tests were conducted between the 1st of April and the 11th 2025. The OONI probe App was also ran on occasion on Windows and mobile devices, though this test suite is far less comprehensive in comparison to the CLI. 

It is important to recognize that, particularly in the case of Israel, different individuals may experience censorship of the Internet to varying degrees. The research focuses on replicating the experience of the average Israeli living in Tel Aviv. Having previously discussed the Gaza Strip, it is clear that the user experience varies greatly between the two areas. Further research may look at comparing Internet censorship experienced internally in Israel, but that is outside the scope of this work.

In the following section, the relevant context regarding the network conditions while gathering ground truth will be discussed. Having previously highlighted the varying levels at which internet censorship can be conducted, let us now define the specific network characteristics under which the OONI probe was run for both countries.

\subsection{Network Environment Context: Ireland}
As mentioned, the OONI CLI probe is running on a Raspberry Pi 5 connected to a home residential network over WiFi. The provider is Virgin Media (AS12388), registered under Liberty Global B.V. ISP (AS6830). 

Details regarding the operating system flashed to the Raspberry Pi and packages required for this are illustrated below in the 'Guide to Replicating Results.' 

\subsection{Network Environment Context: Israel}
Ground truth gathering in Israel was done using a virtual machine as described in the section on methodology. The virtual machine is operates within a data center hosted by O.M.C. COMPUTERS \& COMMUNICATIONS LTD (AS44709), downstream of its owner Kamatera Inc. (AS36007). This differs from the residential network that is being tested in Ireland.


\section{Website Connectivity Tests}
\subsection{Public OONI Database}
\subsection{Ground Truth via SSH}
\subsection{my-websites.txt}
This file contains the additional 170 website connectivity tests I wished to conduct. The list was compiled with Griff Steinman, and additional tests relating to Israel were added. The contents can be found within the Appendix section. Below is a summary of the results from the daily running of this additional web connectivity test suite.

\begin{table}[H]
\centering
\caption{Summary of Blocked vs. Unblocked Websites by Country}
\begin{tabular}{lcc}
\toprule
\textbf{Country} & \textbf{Unblocked} & \textbf{Blocked} \\
\midrule
Ireland & 151 & 19 \\
Israel    & 155 & 15 \\   
\bottomrule
\end{tabular}
\label{tab:blocked_summary}
\end{table}

The following figure illustrates the blocking methods for the additional web connectivity tests. 

\begin{table}[H]
\centering
\caption{Distribution of Blocking Methods Detected in Ireland \&Israel}
\begin{tabular}{lcc}
\toprule
\textbf{Blocking Method} & \textbf{Ireland} & \textbf{Israel} \\
\midrule
TCP/IP         & 15 & 13 \\
DNS            & 1  & 1  \\
HTTP           & 1  & 1  \\
Error/Failure  & 3  & 0  \\
\bottomrule
\end{tabular}
\label{tab:blocking_methods_comparison}
\end{table}

\section*{Blocked Websites by Method}



\begin{tabular}{ll} 
\caption{Ireland}
\toprule
\textbf{Blocking Method} & \textbf{Websites} \\
\midrule
TCP/IP & \url{http://www.socom.mil} \\
       & \url{https://www.mp3.com} \\
       & \url{http://www.bittornado.com} \\
       & \url{https://libgen.life} \\
       & \url{http://www.oic-oci.org} \\
       & \url{https://doh.centraleu.pi-dns.com/dns-query?...} \\
       & \url{https://im0-tub-com.yandex.net/...} \\
       & \url{https://www.xroxy.com} \\
       & \url{http://www.queernet.org} \\
       & \url{https://www.gamku.com} \\
       & \url{https://www.onlinearabiccasino.com} \\
       & \url{https://www.iasj.net} \\
       & \url{https://www.shabakaty.com} \\
       & \url{http://alhikmae.com} \\
\midrule
DNS    & \url{https://thepiratebay.org} \\
       & \url{https://kickasstorrents.to} \\
       & \url{https://www.iconnecthere.com} \\
       & \url{https://www.rt.com} \\
\midrule
HTTP   & \url{http://www.euthanasia.cc} \\
\bottomrule
\end{tabular}

\vspace{1cm}

\textbf{Israel}

\begin{tabular}{ll} [H]
\toprule
\textbf{Blocking Method} & \textbf{Websites} \\
\midrule
TCP/IP & \url{http://www.socom.mil} \\
       & \url{https://www.mp3.com} \\
       & \url{http://www.bittornado.com} \\
       & \url{https://libgen.life} \\
       & \url{http://www.oic-oci.org} \\
       & \url{https://doh.centraleu.pi-dns.com/dns-query?dns=q80BAAABAAAAAAAAA3d3dwdleGFtcGxlA2NvbQAAAQAB} \\
       & \url{https://im0-tub-com.yandex.net/i?id=462f375c96139f1e41d919b44be2d780\&n=13\&exp=1} \\
       & \url{https://www.xroxy.com} \\
       & \url{http://www.queernet.org} \\
       & \url{https://www.onlinearabiccasino.com} \\
       & \url{https://www.iasj.net} \\
       & \url{http://alhikmae.com} \\
       & \url{https://www.shabakaty.com} \\
\midrule
DNS    & \url{https://www.gamku.com} \\
\midrule
HTTP   & \url{http://www.euthanasia.cc} \\
\bottomrule
\end{tabular}



\subsection{Investigating Aljazeera.com}
Upon initial testing of Aljazeera blocking in Israel, it was surprising to see no evidence of blocking. To investigate further it was decided that the content of the site should be examined and a variety of URLs belonging to it should be tested. Particular attention was given to articles that were critical of Israel.

The file sample\_aljazeeraurls.txt contains 100 URLs for articles hosted on https://aljazeera.com. Below are some helpful commands used to obtain the sitemap and extract URLs for testing. Wget was used to examine the contents of the web page, and the following was observed.

\begin{figure} [H]
    \centering
    \includegraphics[width=1\linewidth]{wgetAljazeera.png}
    \caption{wget showing aljazeera.com content}
    \label{fig:enter-label}
\end{figure}


As seen above, I was able to see unblocked content from the site. Within Appendix, one can find a sample of articles being tested. 


\section{Circumvention Tests}
\subsection{Public OONI Database}
\subsection{Ground Truth via SSH}

\section{Instant Messaging Tests}
\subsection{Public OONI Database}
\subsection{Ground Truth via SSH}

\section{Middlebox Tests}
\subsection{Public OONI Database}
\subsection{Ground Truth via SSH}

\section{Comparitive Analysis: Ireland vs. Israel}
\subsection{Summary: Irish Internet Censorship}
\subsection{Summary: Israeli Internet Censorship}
\subsection{}

\section{Guide to Replicating Results}

\subsection{Investigating Aljazeera.com}
Below are some commands I found helpful while investigating aljazeera.com.

\begin{figure}[H]
    \centering
    \includegraphics[width=1\linewidth]{AljazeeraURLs1.png}
    \caption{Using curl and grep to scrape URLs for articles}
    \label{fig:enter-label}
\end{figure}

\begin{figure}[H]
    \centering
\includegraphics[width=1\linewidth]{AljazeeraComs2.png}
    \caption{Inserting a dash after each URL, can now be ran with OONI}
    \label{fig:enter-label}
\end{figure}


\subsection{Raspberry Pi Setup}
\textbf{Operating System}
The operating system used was the latest available version of Raspberry Pi OS (64-bit) at the time of testing. It was flashed using the Raspberry Pi Imager application over USB. 

\begin{flushleft}
\hspace{1em}\textbf{Raspberry Pi OS (64-bit)}\\[0.5em]
\hspace{1em}Release date: November 19th 2024\\[0.5em]
\hspace{1em}System: 64-bit\\[0.5em]
\hspace{1em}Kernel version: 6.6\\[0.5em]
\hspace{1em}Debian version: 12 (bookworm)
\end{flushleft}


\textbf{Packages Installed}
In order to install the OONI probe CLI, the guide 'Install OONI Probe CLI on Debian/Ubuntu Linux' \cite{ooni-cli-install} was followed. The process was similar to that of installing on the Israeli VM. One obstacle faced in both instances were OpenPGP errors. To solve, the permissions for OONI probe had to be updated to bypass PGP signature verification. To do this, /etc/apt/sources.list.d/ooniprobe.list was edited to include '[trusted=yes]'

\begin{figure} [H]
    \centering
    \includegraphics[width=0.75\linewidth]{PGPERROR.png}
    \caption{PGP Error installing CLI}
    \label{fig:enter-label}
\end{figure}


