\chapter{Literature Review}

\section{Introduction}

Write an introduction outlining the reasons for lit review... Briefly elude to some high level differences between Ireland and Israel.

The purpose of this literature review is to survey and consider published work regarding internet censorship globally, with a particular focus on that of Ireland and Israel. Legislation, important events and other notable areas will be discussed and compared in order to gain a greater understanding the differences between internet censorship experienced in Israel and Ireland. Internet censorship is constantly evolving as it competes with privacy based tools in an 'Arms race' of sorts. This makes researching and understanding how censors achieve their purpose of particular importance. To properly understand the current situations faced by both Israeli and Irish citizens using the internet, a broad analysis of existing literature and ongoing research had to be considered. This section lays the groundwork for the thesis, detailing how both countries approach to internet censorship has evolved over the years. 

Internet usage is rising year on year globally as more users are free to surf the web. Our World in Data, an independent organisation that tracks internet usage statistics suggests that as of 2023, 67.4\% of the world was connected to the internet. This is a staggering number of individuals that is only set to increase. With more people relying on the internet for their livelihood, communication or otherwise, internet censorship is becoming a more pressing matter. It has also been noted previously that, based on OONI data, censorship is rising globally. This growth highlights the need for transparency and regulation surrounding user rights and privacy. 

As previously mentioned, a common misconception about the internet is that it's content is not manipulated. Another misconception held by many is that internet censorship occurs in few countries. This is also false, with censors increasing their restrictions continuously. According to Bischoff in his online article mapping internet censorship and geographies "This year we saw nearly 60 countries increase their internet censorship in some way, compared to 50 from last year’s study."\cite{bischoff2025internet} This is a troubling reality as internet content is increasingly being censored not just in authoritarian countries but by democratic states. 

\section{Literature Review Methodology}

In conducting this literature review, sources from a variety of mediums were used. Research was conducted primarily using the internet, focusing on academic papers and peer reviewed literature. Other sources like Trinity College Library, online articles and journals were also considered and sources were cross checked with relevant authorities. Information regarding country-specific legislation was taken from the official state-run website of those countries. Sources deemed potentially unreliable were placed under a higher level of scrutiny. It is worth mentioning that researching this area can be difficult as state level censorship is typically clandestine and overt. 



\section{Findings of the Literature Review}
\subsection{Ireland Historically }


\subsection{Ireland Today}





\subsection{Israel Historically}
Since the early 2000s, internet access has become increasingly available in Israel. In a paper discussing internet usage in Israel, Fisher speaks of "an increase of 152\% in the number of Israeli households connected to the Internet during the period 2000–2005." \cite{FISHER2006984} 
Being one of the early adopters of internet infrastructure, it is not surprising to learn that Israel is an active censor online. 

In discussing public reaction to censorship of the media and the Israeli Defense Forces' capabilities, it is important to understand the history behind Israel's notorious state security operations. 
"the IDF is highly trusted by a society that deeply values the defense system, it is very difficult to criticize its deficiencies" \cite{MAGEN2018287}

It is also important to explore Israel's national security concerns and wars, as they have a large influence on media censorship. Historically, the Israeli Defense Forces (IDF) have had to answer for media censorship through their Spokesperson Unit (ISU). 

"the ISU is continually fluctuating between openness and opaqueness because its activities are affected by so many internal and external factors" \cite{MAGEN2018287}


In following global trends, Israeli internet censorship has developed over the years to cover a variety of security and political concerns. Israel entrusts its internet censorship operation to the Israeli Military Censor. This group is primarily responsible for the protection of Israeli security interests, and headed by Israel's minister of defense, currently Israel Katz.\cite{MOD_Israel} 

According to the Internet Monitor, a data analysis and collection tool used to monitor internet access and online content controls states about Israel's freedom of press: "Modern censorship of information operates through voluntary agreements between the military and the Israeli Committee of Daily Newspaper Editors. Even though these agreements lack full consent from media in the country, all media organizations operating in Israel must abide by the censor's decisions." \cite{internet_monitor_israel}



\subsection{Israel Today}
Today, a large majority of Israeli citizens have access to the internet. 
DataReportal, a website responsible for collecting and publishing global digital reports states: "there were 8.51 million internet users in Israel at the start of 2024, when internet penetration stood at 92.1 percent." \cite{Digital2024Israel}

Reporters Without Borders, responsible for the World Press Freedom Index have ranked Israel as 101st in the world as of 2024. This ranking is based on the level of freedom enjoyed by journalists and media. "Press freedom is defined as the ability of journalists as individuals and collectives to select, produce, and disseminate news in the public interest independent of political, economic, legal, and social interference and in the absence of threats to their physical and mental safety." \cite{rsf_israel}

"In June 2017, after a few years of no blocking, the Palestinian Authority ordered ISPs to block 12 news websites affiliated with the rival Islamist group Hamas which controls the Gaza Strip, websites affiliated with dismissed Fatah leader Mohammed Dahlan, and 10 news websites that provide news and views on Palestinian politics." \cite{zittrain2017shifting}


\section{Analysis: Ireland vs Israel}


\subsection{Similarities}



\subsection{Differences}



\section{Conclusions}

